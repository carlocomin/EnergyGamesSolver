\section{Additional Tools}
\label{sec:tools}

\subsection{Obfuscator}

This tool randomly permutes the identifiers of nodes as well as the order
of nodes' successors in a parity game. This is useful to confuse algorithms
which perform very well on certain benchmarks because of the order in which
nodes and successors are given.

\begin{example}
Take the ladder game of index 4, as introduced in Sect.~\ref{sec:laddergame}
below. In \pgsolver's specification format it looks like this.
\begin{verbatim}
    ~/pgsolver> bin/laddergame 4
    parity 7;
    0 0 0 1,2;
    1 1 1 2,3;
    2 0 0 3,4;
    3 1 1 4,5;
    4 0 0 5,6;
    5 1 1 6,7;
    6 0 0 7,0;
    7 1 1 0,1;
\end{verbatim}
Each player owns 4 out of the eight nodes. Since all nodes have outdegree 2, there
are $2^4 = 16$ strategies for each of the player. In general, there are $2^n$
different strategies in the game of index $n$ but only one of them is
a winning strategy. However, the winning strategy for player $0$ for example is
an obvious one in this case. It consists of always choosing the first node in the
list of successors. Equally, the winning strategy for player $1$ consists of always
choosing the last node in each list. This is hardly a good hiding of the winning
strategies.

Obfuscator employs a random generator to destroy the order of the successors and
the order in which the nodes are given. A typical obfuscation of the ladder game of
index 4 would be the following.
\begin{verbatim}
    ~/pgsolver> bin/laddergame 4 | bin/obfuscator
    parity 7;
    0 1 1 6,1;
    1 1 1 7,5;
    2 1 1 3,0;
    3 0 0 0,6;
    4 0 0 2,3;
    5 1 1 4,2;
    6 0 0 1,7;
    7 0 0 5,4;
\end{verbatim}
\end{example}

Obfuscator is automatically built through \verb#make obfuscator#,
\verb#make tools#, or \verb#make all#.
It takes a parity game on \texttt{stdin} in the specification format described above
and returns the permuted game on \texttt{stdout}. The currently understood command-line
parameters are:
\begin{description}
\item[{\tt --disablenodeobfuscation}] \enspace or {\tt -dn} \\
	Disables the obfuscation of the ordering of nodes.

\item[{\tt --disableedgeobfuscation}] \enspace or {\tt -de} \\
    Disables the obfuscation of the ordering of the edges.
\end{description}



\subsection{Compressor}

This tool compresses a given parity game s.t.\ the winning sets and strategies of
the resulting game equal those of the original game.

\begin{example}
Take the Jurdzi{\'n}ski game of $J_{1, 2}$, as introduced in Sect.~\ref{sec:jurdgame} below.
In \pgsolver's specification format it looks like this.
\begin{verbatim}
    ~/pgsolver> bin/jurdzinskigame 1 2
    parity 11;
    0 1 0 5;
    2 1 0 7;
    4 2 1 5,0;
    6 2 1 5,7,2;
    5 2 0 4,6,9;
    7 2 0 7,6;
    8 4 0 9;
    10 4 0 9,11;
    9 3 1 8,10,5;
    11 3 1 10,11,7;
\end{verbatim}
It is possible to push priorities along edges overriding smaller priorities. It is also possible
to reassign priorities just keeping parities and the order inbetween them. This leads to the
following game:
\begin{verbatim}
    ~/pgsolver> bin/jurdzinskigame 1 2 | bin/compressor -pr -pp
    parity 11;
    0 0 0 5;
    2 0 0 7;
    4 0 1 5,0;
    5 0 0 4,6,9;
    6 0 1 5,7,2;
    7 0 0 7,6;
    8 2 0 9;
    9 1 1 8,10,5;
    10 2 0 9,11;
    11 1 1 10,11,7;
\end{verbatim}
It is also possible to compress the original game's identifier space eliminating unused node indetifiers.
\begin{verbatim}
    ~/pgsolver> bin/jurdzinskigame 1 2 | bin/compressor --nodes
    parity 9;
    0 1 0 3;
    1 1 0 5;
    2 2 1 3,0;
    3 2 0 2,4,7;
    4 2 1 3,5,1;
    5 2 0 5,4;
    6 4 0 7;
    7 3 1 6,8,3;
    8 4 0 7,9;
    9 3 1 8,9,5;
\end{verbatim}
Note that the original game does not possess a node called $1$ for instance.
\end{example}

Compressor is automatically built through \verb#make compressor#,
\verb#make tools#, or \verb#make all#. It takes a parity game in the specification format described above
on \texttt{stdin} or from a given file, and returns the compressed game on \texttt{stdout}. The currently
understood command-line parameters are:
\begin{description}
\item[\nonterminal{filename}] \ \\
   Tells the tool to look for the specification of the parity game to transform in the file
   \nonterminal{filename}.

\item[{\tt --priorities}] \enspace or {\tt -pr} \\
    It reduces the priority of each node preserving their parities as well as the relation $\le$
    among priorities of nodes within an SCC of the game graph.

\item[{\tt --priopropagation}] \enspace or {\tt -pp} \\
    It tries to reduce the overall number of priorities occurring in the game by pushing priorities
    along edges of the game graph overriding smaller ones. If all nodes leading to a node $v$ have a
    greater priority than the node itself, the priority of this node is altered to the least predecessing
    priority. Similarly the priority of a node is adapted if all successors of a node have a greater
    priority than the node itself. This process is iterated until stability is reached.

    Obviously, this process may destroy the preservation of parities and the less-or-equals relation among
    priorites described above.

\item[{\tt --antipropagation}] \enspace or {\tt -ap} \\
	It tries to set the priority of as many nodes as possible to zero. If all nodes leading to a node
	$v$ have a greater priority than the node itself, the priority of this node can be set to zero.
	Similarly the priority of a node is adapted if all successors of a node have a greater priority than
	the node itself. This process is iterated until stability is reached.

\item[{\tt  --fakealt}] \enspace or {\tt -fa} \\
    Keep fake alternation w.r.t.\ priorities. If this option is enabled the compaction of priorities
    never assigns the same priority to nodes that had different priorities before, even if there is
    no other priority occurring inbetween and both priorities in question are of the same parity.

\item[{\tt --wholegame}] \enspace or {\tt -pp} \\
    The compression of priorities is usually done independently for each SCC of the game graph.
    This switch causes the graph's SCC decomposition to be ignored. This will in general result
    in a worse compression rate.

\item[{\tt --nodes}] \enspace or {\tt -no} \\
    Performs a compression of the space of identifiers of a parity game, resulting in a possible
    down-shift of identifiers.

\item[{\tt --minmaxswap}] \enspace or {\tt -mm} \\
    Transforms a min-parity game into a max-parity one and vice versa.
\end{description}


\subsection{Combinator}

This tool combines two or more parity games into one single parity game by shifting the node numbers.
Neither additional edges between the games are added nor other modifications are done.

\begin{example}
We just create two small random games and combine them to see what happens.
\begin{verbatim}
    ~/pgsolver> bin/randomgame 5 5 2 3 | tee game1.gm
    parity 4;
    0 1 1 3,4 "0";
    1 4 0 1,4 "1";
    2 5 0 0,2,4 "2";
    3 1 0 0,2,4 "3";
    4 0 1 0,1,3 "4";

    ~/pgsolver> bin/randomgame 5 5 2 3 | tee game2.gm
    parity 4;
    0 4 0 2,3 "0";
    1 5 0 0,4 "1";
    2 4 1 0,2,3 "2";
    3 4 0 2,4 "3";
    4 4 1 0,3 "4";
\end{verbatim}
A combination of both games can be achieved as follows:
\begin{verbatim}
    ~/pgsolver> bin/combine game1.gm game2.gm
    parity 9;
    0 1 1 3,4 "0";
    1 4 0 1,4 "1";
    2 5 0 0,2,4 "2";
    3 1 0 0,2,4 "3";
    4 0 1 0,1,3 "4";
    5 4 0 7,8 "0";
    6 5 0 5,9 "1";
    7 4 1 5,7,8 "2";
    8 4 0 7,9 "3";
    9 4 1 5,8 "4";
\end{verbatim}
\end{example}

Combine is automatically built through \verb#make combine#,
\verb#make tools#, or \verb#make all#.
It takes arbitrarly many parity games as command-line parameters and returns the
combined game on \texttt{stdout}. There are no other command-line parameters.


\subsection{Transformer}

This tool transforms a given parity game into an (somehow) equivalent parity game fulfilling
certain properties -- depending on the user's command line specification. The transformed
parity game can be associated with the original game s.t.\ winning sets and strategies can
be easily back-transformed.

% Additionally, winning strategies in the transformed game can be canonically identified with winning strategies in the original game: the totality
% transformation essentially connects nodes with out-degree 0 with themselves and the choice-alternation transformation replaces all edges between
% nodes belonging to the same player with dummy transfer nodes belonging to the other player.

\begin{example}
Again, take the ladder game of index 3, as introduced in Sect.~\ref{sec:laddergame} below.
In \pgsolver's specification format it looks like this.
\begin{verbatim}
    ~/pgsolver> bin/laddergame 3
    parity 5;
    0 0 0 1,2;
    1 1 1 2,3;
    2 0 0 3,4;
    3 1 1 4,5;
    4 0 0 5,0;
    5 1 1 0,1;
\end{verbatim}
The game is obviously total but not choice-alternating as node $0$ is directly connected to node $2$
with both of them belonging to player $0$. A choice-alternation version of this game can obtained as follows:
\begin{verbatim}
    ~/pgsolver> bin/laddergame 3 | bin/transformer --alternating
    parity 11;
    0 0 0 1,6;
    1 1 1 2,7;
    2 0 0 3,8;
    3 1 1 4,9;
    4 0 0 5,10;
    5 1 1 0,11;
    6 0 1 2;
    7 0 0 3;
    8 0 1 4;
    9 0 0 5;
    10 0 1 0;
    11 0 0 1;
\end{verbatim}
\end{example}

Transformer is automatically built through \verb#make transformer#,
\verb#make tools#, or \verb#make all#.
It takes a parity game in the specification format described above on \texttt{stdin} or from a file,
and returns the transformed game on \texttt{stdout}. The currently understood command-line parameters are:

\begin{description}
\itemsep3mm
\item[\nonterminal{filename}] \ \\
   Tells the tool to look for the specification of the parity game to transform in the file
   \nonterminal{filename}.

\item[{\tt --total}] \enspace or {\tt -to} \\
   Perform a totality transformation on the given parity game.

\item[{\tt --alternating}] \enspace or {\tt -al} \\
   Perform a choice-alternation transformation on the given parity game.

 \item[{\tt --singlescc}] \enspace or {\tt -scc} \\
   Perform a simple transformation that results in a single-scc-parity game. The winning sets and
   strategies of the original game and the transformed game match.

 \item[{\tt --prioalignment}] \enspace or {\tt -pa} \\
   Performs a transformation s.t.\ the parity of each node of the resulting game matches its player
   unless its priority is zero.

\item[{\tt --dummynodes}] \enspace or {\tt -dn} \\
   Performs a transformation that divides each edge by an additional node with priority $0$.

\item[{\tt --uniquizeprios}] \enspace or {\tt -up} \\
   Performs a transformation of the occurring priorities s.t.\ every occurring priority occurs only once.

 \item[{\tt --cheapescapecycles}] \enspace or {\tt -ce} \\
   Adds two low-priority cycles $c_0$ and $c_1$ to the game that are profitable for either one of the
   player ($c_0$ for player $0$ and $c_1$ for player $1$). All nodes in the original belonging to
   player $0$ get additional edges leading to $c_1$ and all nodes belonging to player $1$ get
   additional edges leading to $c_0$.

\item[{\tt --bouncingnode}] \enspace or {\tt -bn} \\
   Replaces every self-cycle by a bouncing node.

 \item[{\tt --increasepriorityoccurrence}] \enspace or {\tt -ip} \\
   Increases the occurrence of all priorities by one by simply inserting a long cycle that is not
   connected to the rest of the game.

\item[{\tt --antiprioritycompactation}] \enspace or {\tt -ap} \\
   Adds additional nodes that prohibits any priority compactation.
\end{description}

Note that the command-line parameters that are specified are carried out in the same ordering as they
appear in the call of the transformer tool. It is also possible to specify a parameter more than once.


\subsection{Benchmark Tool}

This tool helps the user to carry out benchmarks and to compare the different solvers with each other. 
You can specify a list of games that are to be benchmarked with a list of solver that also has to be 
specified.

You can either choose to receive a verbose output showing the exact timing of each case or a simple line 
containing data that can be easily parsed by \texttt{gnuplot}. The latter output option is usually to be 
used in combination with a shell script creating a whole benchmark series.

\begin{example}
The following example shows how to benchmark the recursive and the strategy improvement algorithm on a 
random game with 1000 nodes. To smooth multitasking interference factors, each test case is carried out 
3 times.
\begin{verbatim}
    ~/pgsolver> bin/randomgame 1000 1000 2 4 | bin/benchmark -re -si -t 3
    Benchmarking stratimprove...
    Game # 0, Iteration #1...0.02 sec
    Game # 0, Iteration #2...0.02 sec
    Game # 0, Iteration #3...0.02 sec
    Finished. Best: 0.02 sec  Avg: 0.02 sec  Worst: 0.02 sec

    Benchmarking recursive...
    Game # 0, Iteration #1...0.01 sec
    Game # 0, Iteration #2...0.00 sec
    Game # 0, Iteration #3...0.01 sec
    Finished. Best: 0.00 sec  Avg: 0.01 sec  Worst: 0.01 sec

    +-----------------------------------------------------------+
    | Benchmark Statistics                                      |
    +-----------------------------------------------------------+
    | Solver       |         Best |      Average |        Worst |
    +-----------------------------------------------------------+
    | recursive    |     0.00 sec |     0.01 sec |     0.01 sec |
    | stratimprove |     0.02 sec |     0.02 sec |     0.02 sec |
    +-----------------------------------------------------------+
\end{verbatim}
\end{example}

Benchmark is automatically built through \verb#make benchmark#, \verb#make tools#, or \verb#make all#.
It takes one ore more parity games in the specification format described above on \texttt{stdin} or
from files, and carries out the benchmark. The currently understood command-line parameters are:

\begin{description}
\itemsep3mm
\item[\nonterminal{filename}] \ \\
   Tells the tool to look for the specification of the parity game in the file
   \nonterminal{filename}.

\item[{\tt --all}] \enspace or {\tt -a} \\
   Benchmark all available parity games solvers.

\item[{\tt --\nonterminal{solver}}] \ \\
   Benchmark the specified solver. All solvers that are compiled into \pgsolver are available here.

\item[{\tt --silent}] \enspace or {\tt -s} \\
   Only print the final statistics on \texttt{stdout}.

\item[{\tt --gnuplotformat}] \enspace or {\tt -gp} \\
   Only print a gnuplot-compatible line.

 \item[{\tt --timeout \nonterminal{timeout}}] \enspace or {\tt -t \nonterminal{timeout}} \\
   If one of the solvers requires more than $t$ seconds on one of the test cases, the solver is ignored
   for the rest of the series. There is no timeout by default.

\item[{\tt --name \nonterminal{title}}] \enspace or {\tt -n \nonterminal{title}} \\
   Specifies the title of the benchmark that is to be printed in the final statistics.

 \item[{\tt --times \nonterminal{n}}] \enspace or {\tt -t \nonterminal{n}} \\
   Specifies the number of iterations a single test case is carried out per algorithm. By default, $n$
   is 10.

\item[{\tt --\nonterminal{optimisation-option}}] \ \\
   All optimisation options that are available to \texttt{pgsolver} are also available here.
\end{description}



%%% Local Variables:
%%% mode: latex
%%% TeX-master: "main"
%%% End:
